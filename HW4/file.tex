\documentclass[11pt]{article}
\author{Lawrence Liu}
\usepackage{subcaption}
\usepackage{graphicx}
\usepackage{amsmath,amssymb,stmaryrd}
\usepackage{physics}
\usepackage{pdfpages}
\newcommand{\Laplace}{\mathscr{L}}
\setlength{\parskip}{\baselineskip}%
\setlength{\parindent}{0pt}%
\usepackage{xcolor}
\usepackage{listings}
\definecolor{backcolour}{rgb}{0.95,0.95,0.92}
\usepackage{amssymb}
\lstdefinestyle{mystyle}{
    backgroundcolor=\color{backcolour}}
\lstset{style=mystyle}
\title{ECE 133B HW4}
\begin{document}
\maketitle
\section*{Problem 1}
\subsection*{(a)}
If we represent the group that column $a_j$ is assigned to as a one hot vector 
$C_j$, and we have the a matrix $B$ with column $i$ 
corresponding to the $i$th centroid then we have:
$$J^{\text{clust}}=\frac{1}{n}\sum_{j=1}^{n}||a_j-BC_j||^2$$
When we want to minimize, we can ignore the constant $\frac{1}{n}$ and we have that:
we want to minimize:
$$\sum_{j=1}^{n}||a_j-BC_j||^2$$
We can represnt this in matrix from with $C$ being a $k\times n$ matrix with one hot 
encoding for the columns. We can do this since if $C$ has columns that are one hot encodings, $BC$ effectively 
selects the column of $B$ based on the index of the one hot encoding of $C$. Therefore we can see
that the minimization problem is equivalent to:
$$\min_{C,B}||A-BC||_{F}^2$$
With the constraint that $C$ is a one hot encoding matrix.
\subsection*{(b)}
We have that 
\begin{align*}
    ||A-BC||_F^2 &= \trace((A-BC)^T(A-BC))\\
    &= \trace(A^TA-C^TB^TA-A^TBC+C^TB^TBC)\\
    &= \trace(A^TA)-\trace(C^TB^TA)-\trace(A^TBC)+\trace(C^TB^TBC)\\
    &= \trace(A^TA)-2\trace(A^TBC)+\trace(C^TB^TBC)
\end{align*}
If we take the derivative we get that we want:
\begin{align*}
    \frac{\partial}{\partial C}\left(-2\trace(A^TBC)+\trace(C^TB^TBC)\right) &=0\\
    -2B^TA+2B^TBC&=0\\
    B^TBC&=B^TA
\end{align*}
Because $B$ is not necissarily a full rank matrix, we cannot simply calculate the 
Moore-Penrose inverse. Rather we must SVD decompose $B=U\Sigma V^T$ and then we have that:
$$C = V\Sigma^{-1}U^TA$$
Where $\Sigma^{-1}$ simply denotes taking the reciprocal of the diagonal entries of $\Sigma$ and then 
taking the transpose of the resulting matrix, this takes $k$ operations.\\\\
We have that $U^T$ is of size $m\times m$ and $A$ is of size $m\times n$, and 
$\Sigma^{-1}$ is of size $k\times m$, and $V$ is of size $k\times k$. Therefore we have that
the multiplication $U^TA$ takes $2nm^2$ operations. Since the diagonal of $\Sigma^{-1}$ is
the only nonzero part of $\Sigma^{-1}$ we have that the multiplication $\Sigma^{-1}U^TA$ would
simply takes only $km$ operations. Finally we have that the multiplication $V\Sigma^{-1}U^TA$
takes $2nk^2$ operations.\\\\
We also have that the SVD decomposition takes on the order of $mk^2$ operations. Therefore we have that 
the complexity of the algorithm is $\boxed{O(nm^2)}$ assuming that $k<m$ and $k<n$.
\subsection*{(c)}
We may want to use the rank-k optimization if we prioritize the speed 
over the accuracy of the clustering. Also with the rank-k optimization we can 
get weights that could be translated to some kind of "confidence" or "probability" of 
the clustering.
\section*{Problem 2}
\subsection*{(a)}
Let $A = U_1\Sigma_1 V_1^T$ and $B = U_2\Sigma_2 V_2^T$ be the SVDs of $A$ and $B$ respectively.
Then we have that:
$$AA^T = U_1\Sigma_1^2 U_1^T$$
$$BB^T = U_2\Sigma_2^2 U_2^T$$
Since $V_2^TV_2=I$ and $V_1^TV_1=I$. We can see that these are the eigendecompositions of 
$AA^T$ and $BB^T$ respectively. Therefore we can see that 
the eigenvalues of $AA^T$ and $BB^T$ are $\Sigma_1^2$ and $\Sigma_2^2$ respectively. Likewise
the eigenvectors of $AA^T$ and $BB^T$ are $U_1$ and $U_2$ respectively. Therefore we can see that
if 
$$AA^T = BB^T$$
Then we have that $A$ and $B$ must have the same singular values and the same left 
singular vectors.
\subsection*{(b)}
We have that 
$$A^TB = V_1 \Sigma^T U^T U \Sigma V_2^T$$
$$A^TB = V_1 \Sigma^T \Sigma V_2^T$$
We can see that 
$$QH= V_1V_2^T V_2 \Sigma^T \Sigma V_2^T$$
$$QH = V_1 \Sigma^T \Sigma V_2^T$$
So we have
$$A^TB = QH$$
\subsection*{(c)}
\begin{align*}
    AQ &= U\Sigma V_1^T V_1 V_2^T\\
    AQ &= U\Sigma V_2^T\\
    AQ &= B
\end{align*}
\section*{Problem 3}
\subsection*{(a)}
The laplacian $L$ of the graph would be:
$$L=\begin{bmatrix}
    (n-1) & -1 & \dots & -1\\
    -1 & (n-1) & \dots & -1\\
    \vdots & \vdots & \ddots & \vdots\\
    -1 & -1 & \dots & (n-1)
\end{bmatrix}$$
\subsection*{(b)}
We have that 
$$L1 = 0$$
Therefore $0$ is an eignevalue of $L$, and since the rank of 
$L$ is $n-1$, we have that $0$ is an eigenvalue of $L$ with multiplicity $1$.\\
We can decompose $L$ as
$$L =(n-1)I-A$$
Where is a matrix with all ones except for the main diagonal. We can see that 
$$A+I = 11^T$$
Therefore we have that $A$ has eigenvalue of $-1$ with multiplicity of 
$n-1$, since $11^T$ has rank of $1$ and therefore has nullity of $n-1$. Therefore we have that 
$L$ has eigenvalue of $n$ with multiplicity of $n-1$.
\end{document}
