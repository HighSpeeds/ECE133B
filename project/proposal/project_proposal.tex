\documentclass[10pt]{article}
\author{Lawrence Liu}
\usepackage{subcaption}
\usepackage{graphicx}
\usepackage{amsmath,amssymb,stmaryrd}
\usepackage{physics}
\usepackage{pdfpages}
\newcommand{\Laplace}{\mathscr{L}}
\setlength{\parskip}{\baselineskip}%
\setlength{\parindent}{0pt}%
\usepackage{xcolor}
\usepackage{listings}
\definecolor{backcolour}{rgb}{0.95,0.95,0.92}
\usepackage{amssymb}
\lstdefinestyle{mystyle}{
    backgroundcolor=\color{backcolour}}
\lstset{style=mystyle}
\title{ECE 133B Project Proposal}
\begin{document}
\maketitle
Our Project Proposal is using Singular Value Decomposition (SVD) and Kernel 
Density Estimation to optimizing portfolio allocation. Our goal will be to deliver a 
annualized sharpe ratio that is higher than the S\&P500, and the optimal portfolio given by 
Markowitz's Mean-Variance Portfolio Optimization. We will use the S\&P500 dataset
over the past 10 years as the dataset.\\\\
Our approach will be the following, we will use SVD to identify individual sectors of the 
dataset that are correlated. Then we will use Kernel Density Estimation to estimate the 
multivariate probability density function of the returns of each sector and thus get a more accurate for the first and second moments of the 
distribution. Ie  we have that the probability density function 
that the return for each sector $r_1, r_2, \dots, r_n$ is given by
$$
f(r_1, r_2, \dots, r_n) = \frac{1}{m}\sum_{i=1}^m k(r_i')
$$
where $k$ is the kernel function and $r_i'$ is the vector representing the 
return of the $n$ sectors (so a vector of size $n$) for the $i$th observation 
in our trainset of $m$ observations.\\\\
We use Kernel Density Estimation to 
estimate the probability density function of the returns of each sector because they will likely not be 
normally distributed as Markowitz's Mean-Variance Portfolio Optimization assumes. Specifically we will investigate using 
two kernels, a Gaussian Kernel and a Multivariate T Kernel, to estimate the probability density function of the returns of each sector. To optimize
the parameters of the kernels we will use cross validation and gradient descent, for paremeters that are a matrices such as the 
$\Sigma$ for a Gaussian Kerenel we will first decompose it with Cholesky Decomposition and then optimize the parameters of the
decomposed matrix.
\end{document}

% We will investigate two kernels, 
% a Gaussian Kernel and a Multivariate T Kernel, 

% outperform 
% the S\&P500, and the optimal portfolio given by Markowitz's Mean-Variance

% We will use the S&P500 dataset 
% over the past 10 years as the dataset. Then we will use Singular Value Decomposition (SVD)
% to identify sectors of the dataset that are correlated. 