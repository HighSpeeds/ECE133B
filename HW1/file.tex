% \documentclass[11pt]{article}
% \author{Lawrence Liu}
% \usepackage{subcaption}
% \usepackage{graphicx}
% \usepackage{amsmath,amssymb,stmaryrd}
% \usepackage{physics}
% \usepackage{pdfpages}
% \newcommand{\Laplace}{\mathscr{L}}
% \setlength{\parskip}{\baselineskip}%
% \setlength{\parindent}{0pt}%
% \usepackage{xcolor}
% \usepackage{listings}
% \definecolor{backcolour}{rgb}{0.95,0.95,0.92}
% \usepackage{amssymb}
% \lstdefinestyle{mystyle}{
%     backgroundcolor=\color{backcolour}}
% \lstset{style=mystyle}
\documentclass[12pt]{article}
\author{Lawrence Liu}
\usepackage{subcaption}
\usepackage{graphicx}
\usepackage{amsmath}
\usepackage{pdfpages}
\newcommand{\Laplace}{\mathscr{L}}
\setlength{\parskip}{\baselineskip}%
\setlength{\parindent}{0pt}%
\usepackage{xcolor}
\usepackage{listings}
\definecolor{backcolour}{rgb}{0.95,0.95,0.92}
\usepackage{amssymb}
\usepackage{empheq}

\newcommand*\widefbox[1]{\fbox{\hspace{2em}#1\hspace{2em}}}
\lstdefinestyle{mystyle}{
    backgroundcolor=\color{backcolour}}
\lstset{style=mystyle}
\title{ECE 133B HW1}
\begin{document}
\maketitle
\section{Problem 1}
Let $A_i$ denote the ith column of $A$, and $C'_i$ denote the ith column of $C$ that is not 
in the null space of $A$, and $B'_i$ denote the ith column of $B$ that is not in the null space of $A$. We have that for any $w\in W+v$.
$$w = \sum_{i=1}^{n}A_ix_i+\sum_{i=1}^{m}B'_iy_i+\sum_{i=1}^{p}C'_iz_i$$
Therefore we can see that $[A,B,C]$ is a basis for $W+V$. 

\section*{Problem 2}
\subsection*{(a)}
If we want a path to be between any two nodes, we would need at least $n-1$ edges with one connected to each node. With these we could 
form a path between any two nodes. Therefore there will be $n-1$ linearly independent vectors in the node incidence matrix $A$, 
and the other $m-n+1$ vectors will be linear combinations of these. Therefore the rank of $A$ is $n-1$.
\subsection*{(b)}
Then the rank would be $\sum_{i=1}^{n'}(n_i-1)$ where $n'$ is the number of clusters and $n_i$ is the number of nodes in the $i$th cluster
this is because each cluster has at least $n_i-1$ linearly independent vectors in the node incidence matrix $A$, and since these cluster's
are not connected, their vectors are linearly independent. Therefore the rank of $A$ is $\sum_{i=1}^{n'}(n_i-1)$.
\section*{Problem 3}
\subsection*{(a)}
Let $AB=C$, then we have that each column of $C$ is a linear combinations of the coloumns of $A$, ie, the ith 
column of $C$, $C_i$ is:
$$
C_i = \sum_{j=1}^{n}A_{j}B_{ij}
$$
where $A_j$ is the jth column of $A$ and $B_{ij}$ is the element in the ith row and jth column of $B$. Thus we have that:
$$
\text{rank}(AB)\leq \text{rank}(A)
$$
Likewise if we transpose we get that 
$$
\text{rank}((AB)^T)=\text{rank}(B^TA^T)\leq \text{rank}(B^T)
$$
Since row rank is the same as column rank, we have that:
$$
\text{rank}(AB)\leq \text{rank}(B)
$$
Therefore we have that:
$$
\text{rank}(AB)\leq \min\left[\text{rank}(A), \text{rank}(B)\right] 
$$
\subsection*{(b)}
We have that the columns of $A+B$ are linear combinations of the columns of $A$ and $B$, therefore they are drawn from 
the combined vector space of the span of $A$ and $B$. Therefore we have that:
$$
\text{rank}(A+B)\leq \text{rank}(A)+\text{rank}(B)
$$
\section*{Problem 4}
We have that 
$$A^2=A$$
$$BCBC=BC$$
$$B^{\dagger}BCBCC^{\dagger}=B^{\dagger}BCC^{\dagger}$$
$$IBCI=I$$
$$BC=I$$
We can write the trace of BC as
$$
\text{tr}(BC)=\sum_{i=1}^{n}\sum_{i=1}^{m}B_{ij}C_{ij}
$$
if B is a (n x m) matrix and C is a (m x n) matrix. We have that 
\begin{align*}
    \text{tr}(BC)&=\sum_{i=1}^{n}\sum_{i=1}^{m}B_{ij}C_{ij}\\
    &=\sum_{i=1}^{m}\sum_{i=1}^{n}C_{ij}B_{ij}\\
    &=\text{tr}(CB)
\end{align*}
Therefore we have that 
$$
\text{tr}(BC)=\text{tr}(CB)=\text{tr}(I)=n
$$
And thus 
$$
\text{tr}(A)=\text{tr}(BC)=n
$$
\end{document}
