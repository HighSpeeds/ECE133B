\documentclass[11pt]{article}
\author{Lawrence Liu}
\usepackage{subcaption}
\usepackage{graphicx}
\usepackage{amsmath,amssymb,stmaryrd}
\usepackage{physics}
\usepackage{pdfpages}
\newcommand{\Laplace}{\mathscr{L}}
\setlength{\parskip}{\baselineskip}%
\setlength{\parindent}{0pt}%
\usepackage{xcolor}
\usepackage{listings}
\definecolor{backcolour}{rgb}{0.95,0.95,0.92}
\usepackage{amssymb}
\lstdefinestyle{mystyle}{
    backgroundcolor=\color{backcolour}}
\lstset{style=mystyle}
\title{ECE 133B HW3}
\begin{document}
\maketitle
\section*{Problem 1}
\subsection*{(a)}
We can write this as:
$$\text{trace}(X^TAX)=\sum_{i=1}^k x_i^T A x_i$$
Where $x_i$ is the $i$th column of $X$. We also have that $x_i^Tx_j = \delta_{ij}$, so we have that 
$x_i,...,x_k$ and the eigenvector corresponding to the
$k$th largest eigenvalues of $A$. Thus, we have that the maximum of 
$$\text{trace}(X^TAX)\geq \sum_{i=1}^k \lambda_i$$
Where $\lambda_i$ is the $i$th largest eigenvalue of $A$.
\subsection*{(b)}
From the same logic we have that $x_i,...,x_k$ are the eigenvectors corresponding to the $k$ smallest eigenvalues of $A$.
 Thus, we have that the minimum of 
$$\text{trace}(X^TAX)\leq \sum_{i=1}^k \lambda_{n+k+1-i}$$
Where $\lambda_i$ is the $i$th largest eigenvalue of $A$.
\subsection*{(c)}
We have that 
\begin{align*}
    X^TAX &= X^TBB^TX\\
    &= (B^TX)^T(B^TX)\\
    &= 