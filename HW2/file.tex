\documentclass[11pt]{article}
\author{Lawrence Liu}
\usepackage{subcaption}
\usepackage{graphicx}
\usepackage{amsmath,amssymb,stmaryrd}
\usepackage{physics}
\usepackage{pdfpages}
\newcommand{\Laplace}{\mathscr{L}}
\setlength{\parskip}{\baselineskip}%
\setlength{\parindent}{0pt}%
\usepackage{xcolor}
\usepackage{listings}
\definecolor{backcolour}{rgb}{0.95,0.95,0.92}
\usepackage{amssymb}
\lstdefinestyle{mystyle}{
    backgroundcolor=\color{backcolour}}
\lstset{style=mystyle}
\title{ECE 133B HW2}
\begin{document}
\maketitle
\section{Problem 1}
We have that 
$$A^{\dagger} = C^T(CC^T)^{-1}(B^TB)^{-1}B^T$$
Therefore we have that 
\begin{align*}
    AA^{\dagger}A &= BC C^T(CC^T)^{-1}(B^TB)^{-1}B^T B C\\
    &=BIIC\\
    &=BC\\
    &=A
\end{align*}
And:
\begin{align*}
    A^{\dagger}AA^{\dagger} &= C^T(CC^T)^{-1}(B^TB)^{-1}B^T B C C^T(CC^T)^{-1}(B^TB)^{-1}B^T\\
    &= C^T(CC^T)^{-1} \left((B^TB)^{-1}(B^T B)\right) \left((C C^T)(CC^T)^{-1}\right) (B^TB)^{-1}B^T\\
    &= C^T(CC^T)^{-1}(B^TB)^{-1}B^T\\
    &= A^{\dagger}
\end{align*}
And 
\begin{align*}
    AA^{\dagger} &= BCC^T(CC^T)^{-1}(B^TB)^{-1}B^T \\
    &= B(B^TB)^{-1}B^T\\
    (AA^{\dagger})^T &=  B(B^TB)^{-1} (CC^T)^{-1} CC^T B^T \\
    &= B(B^TB)^{-1} B^T 
\end{align*}
Therefore we have that $AA^{\dagger}$ is symmetric. Likewise we have 
\begin{align*}
    A^{\dagger}A &= C^T(CC^T)^{-1}(B^TB)^{-1}B^T B C\\
     &= C^T(CC^T)^{-1} C\\
    (A^{\dagger}A)^T = C^T B^T B (B^TB)^{-1} (CC^T)^{-1} C\\
    &= C^T(CC^T)^{-1} C
\end{align*}
Therefore we have that $A^{\dagger}A$ is symmetric.
\section{Problem 2}
\subsection*{(a)}
We have 
\begin{align*}
    A \circ (dd^T) &= \begin{bmatrix}
        A_{11} d_1 d_1 & A_{12} d_1 d_2 & \cdots & A_{1n} d_1 d_n\\
        A_{21} d_2 d_1 & A_{22} d_2 d_2 & \cdots & A_{2n} d_2 d_n\\
        \vdots & \vdots & \ddots & \vdots\\
        A_{n1} d_n d_1 & A_{n2} d_n d_2 & \cdots & A_{nn} d_n d_n\\
    \end{bmatrix}\\
    & = \text{diag} (d) \begin{bmatrix}
        A_{11}d_1 & A_{12}d_2 & \cdots & A_{1n}d_n\\
        A_{21}d_1 & A_{22}d_2 & \cdots & A_{2n}d_n\\
        \vdots & \vdots & \ddots & \vdots\\
        A_{n1}d_1 & A_{n2}d_2 & \cdots & A_{nn}d_n\\
    \end{bmatrix}\\
    &= \text{diag} (d) A \text{diag}(d)
\end{align*}
If we have that $A$ is positive semi definite, then we have that 
$$A=BB^T$$
Therefore we have that 
$$A \circ (dd^T) = \text{diag} (d) BB^T \text{diag}(d)$$
Let $C = \text{diag} (d) B$, then we have that
$$A \circ (dd^T) = CC^T$$
Therefore we have that $A \circ (dd^T)$ is positive semi definite.
\subsection*{(b)}
We have that 
\begin{align*}
    A \circ B & =\begin{bmatrix}
        A_{11} B_{11} & A_{12} B_{12} & \cdots & A_{1n} B_{1n}\\
        A_{21} B_{21} & A_{22} B_{22} & \cdots & A_{2n} B_{2n}\\
        \vdots & \vdots & \ddots & \vdots\\
        A_{n1} B_{n1} & A_{n2} B_{n2} & \cdots & A_{nn} B_{nn}\\
    \end{bmatrix}\\
    &= \begin{bmatrix}
        A_{11} D_{1} D_{1}^T & A_{12} D_{1} D_{2}^T & \cdots & A_{1n} D_{1} D_{n}^T\\
        A_{21} D_{2} D_{1}^T & A_{22} D_{2} D_{2}^T & \cdots & A_{2n} D_{2} D_{n}^T\\
        \vdots & \vdots & \ddots & \vdots\\
        A_{n1} D_{n} D_{1}^T & A_{n2} D_{n} D_{2}^T & \cdots & A_{nn} D_{n} D_{n}^T\\
    \end{bmatrix}\\
    &= \begin{bmatrix}
        C_{1}C_{1}^T D_{1} D_{1}^T & C_{1}C_{2}^T D_{1} D_{2}^T & \cdots & C_{1}C_{n}^T D_{1} D_{n}^T\\
        C_{2}C_{1}^T D_{2} D_{1}^T & C_{2}C_{2}^T D_{2} D_{2}^T & \cdots & C_{2}C_{n}^T D_{2} D_{n}^T\\
        \vdots & \vdots & \ddots & \vdots\\
        C_{n}C_{1}^T D_{n} D_{1}^T & C_{n}C_{2}^T D_{n} D_{2}^T & \cdots & C_{n}C_{n}^T D_{n} D_{n}^T\\
    \end{bmatrix}\\
    &=(C\circ D) (C\circ D)^T
\end{align*}
Where $A=CC^T$ and $B=DD^T$. And $C_{i}$ and $D_{i}$ are the $i$th row of $C$ and $D$ respectively. Therefore we have that $A \circ B$ is positive semi definite.
\subsection*{(c)}
We have that 
\begin{align*}
    v^T (A\circ B)v &= v^T =\begin{bmatrix}
        A_{11} B_{11} & A_{12} B_{12} & \cdots & A_{1n} B_{1n}\\
        A_{21} B_{21} & A_{22} B_{22} & \cdots & A_{2n} B_{2n}\\
        \vdots & \vdots & \ddots & \vdots\\
        A_{n1} B_{n1} & A_{n2} B_{n2} & \cdots & A_{nn} B_{nn}\\
    \end{bmatrix} v \\
    &= v^T\begin{bmatrix}
        \sum_{j=1}^{n} A_{1j} B_{1j} v_j\\
        \sum_{j=1}^{n} A_{2j} B_{2j} v_j\\
        \vdots\\
        \sum_{j=1}^{n} A_{nj} B_{nj} v_j\\
    \end{bmatrix}\\
    &= \sum_{i=1}^{n} \sum_{j=1}^{n} A_{ij} B_{ij} v_i v_j\\
    &= \trace(\text{diag}(v) A \text{diag}(v) B)\\
    &= \trace(\text{diag}(v)^2 AB)
\end{align*}
We have that if $v\neq0$ every element of $\text{diag}(v)^2$
along the diagonal is positive, likewise we have that  $\trace(AB)> 0$, 
therefore we have that $v^T (A\circ B)v >0$ for all $v\neq0$ and therefore $A\circ B$ is positive definite.
\subsection*{(d)}
We have that
$$X_{2} = (X\circ X)$$
Is positive semi definite from part $b$, therefore from induction if we assume 
$$X_{n} = (X\circ X \circ \cdots \circ X)$$
is positive semi definite, then we have that
$$X_{n+1} = (X_{n}\circ X)$$
is positive semi definite from part $b$. Therefore we have that $X_{n}$ is positive semi definite for all $n$. Therefore 
we have that 
$$Y = c_0 I+ \sum_{i=1}^{n} c_i X_{i}$$
is positive semi definite since $c_i$ is nonnegative 
for all $i$ and $X_{i}$ is positive semi definite for all $i$.
\section*{Problem 3}

\end{document}